%!TEX TS-program = pdflatex
%!TEX encoding = utf8
\documentclass[12pt, oneside]{book}
\usepackage[T1]{fontenc}
\usepackage[utf8]{inputenc}
\usepackage[english]{babel}
%% FONTS: libertine+biolinum+stix
\usepackage[mono=false]{libertine}
\usepackage[notext]{stix}

% =====================
% = Datos importantes =
% =====================
% hay que rellenar estos datos y luego
% ir a \begin{document}

\title{Uncertainty treatment of discrete elements in Maxwell equations' resolution using the FDTD method.}
\author{Alberto Prados Pérez}
\date{15 Julio 2021}
\newcommand{\tutores}[1]{\newcommand{\guardatutores}{#1}}
\tutores{Dr. Luis Manuel Díaz Angulo}

% ======================
% = Páginas de títulos =
% ======================
\makeatletter
\edef\maintitle{\@title}
\renewcommand\maketitle{%
  \begin{titlepage}
      \vspace*{1.5cm}
      \parskip=0pt
      \Huge\bfseries
      \begin{center}
          \leavevmode\includegraphics[totalheight=6cm]{sello.jpg}\\[2cm]
          \@title
      \end{center}
      \vspace{1cm}
      \begin{center}
          \@author
      \end{center}
  \end{titlepage}
  
  \begin{titlepage}
  \parindent=0pt
  \begin{flushleft}
  \vspace*{1.5mm}
  \setlength\baselineskip{0pt}
  \setlength\parskip{0mm}
  \begin{center}
      \leavevmode\includegraphics[totalheight=4.5cm]{sello.jpg}
  \end{center}
  \end{flushleft}
  \vspace{1cm}
  \bgroup
  \Large \bfseries
  \begin{center}
  \@title
  \end{center}
  \egroup
  \vspace*{.5cm}
  \begin{center}
  \@author
  \end{center}
  \vspace*{1.8cm}
  \begin{flushright}
  \begin{minipage}{8.45cm}
      Memoria del {\bf Trabajo Fin de Máster}.\\ 
      Máster en Física y Matemáticas (FisyMat) \\ 
      University of Granada.

      \vspace*{7.5mm}

      Tutored by:
      % \vspace*{5mm}
  \end{minipage}\par
  \begin{tabularx}{8.45cm}[b]{@{}l}
      \guardatutores
  \end{tabularx}
   \end{flushright}
      \vspace*{\fill}
   \end{titlepage}
   %%% Esto es necesario...
   \pagestyle{tfg}
   \renewcommand{\chaptermark}[1]{\markright{\thechapter.\space ##1}}
   \renewcommand{\sectionmark}[1]{}
   \renewcommand{\subsectionmark}[1]{}
  }
\makeatother

% ======================================
% = Color de la Universidad de Sevilla =
% ======================================
\usepackage{tikz}
\definecolor{USred}{cmyk}{0,1.00,0.65,0.34}

% =========
% = Otros =
% =========
\usepackage[]{tabularx}
\usepackage[]{enumitem}
\setlist{noitemsep}

% ==========================
% = Matemáticas y teoremas =
% ==========================
\usepackage[]{amsmath}
\usepackage[]{amsthm}
\usepackage[]{mathtools}
\usepackage[]{bm}
\usepackage[]{thmtools}
\usepackage{braket}

\newcommand{\marcador}{\vrule height 10pt depth 2pt width 2pt \hskip .5em\relax}
\newcommand{\cabeceraespecial}{%
    \color{USred}%
    \normalfont\bfseries}
\declaretheoremstyle[
    spaceabove=\medskipamount,
    spacebelow=\medskipamount,
    headfont=\cabeceraespecial\marcador,
    notefont=\cabeceraespecial,
    notebraces={(}{)},
    bodyfont=\normalfont\itshape,
    postheadspace=1em,
    numberwithin=chapter,
    headindent=0pt,
    headpunct={.}
    ]{importante}
\declaretheoremstyle[
    spaceabove=\medskipamount,
    spacebelow=\medskipamount,
    headfont=\normalfont\itshape\color{USred},
    notefont=\normalfont,
    notebraces={(}{)},
    bodyfont=\normalfont,
    postheadspace=1em,
    numberwithin=chapter,
    headindent=0pt,
    headpunct={.}
    ]{normal}
\declaretheoremstyle[
    spaceabove=\medskipamount,
    spacebelow=\medskipamount,
    headfont=\normalfont\itshape\color{USred},
    notefont=\normalfont,
    notebraces={(}{)},
    bodyfont=\normalfont,
    postheadspace=1em,
    headindent=0pt,
    headpunct={.},
    numbered=no,
    qed=\color{USred}\marcador
    ]{demostracion}

% Los nombres de los enunciados. Añade los que necesites.
\declaretheorem[name=Observaci\'on, style=normal]{remark}
\declaretheorem[name=Corolario, style=normal]{corollary}
\declaretheorem[name=Proposici\'on, style=normal]{proposition}
\declaretheorem[name=Lema, style=normal]{lemma}

\declaretheorem[name=Teorema, style=importante]{theorem}
\declaretheorem[name=Definici\'on, style=importante]{definition}

\let\proof=\undefined
\declaretheorem[name=Demostraci\'on, style=demostracion]{proof}


% ============================
% = Composición de la página =
% ============================
\usepackage[
    a4paper,
    textwidth=80ex,
]{geometry}

\linespread{1.069}
\parskip=10pt plus 1pt minus .5pt
\frenchspacing
% \raggedright


% ==============================
% = Composición de los títulos =
% ==============================

\usepackage[explicit]{titlesec}

\newcommand{\hsp}{\hspace{20pt}}
\titleformat{\chapter}[hang]
    {\Huge\sffamily\bfseries}
    {\thechapter\hsp\textcolor{USred}{\vrule width 2pt}\hsp}{0pt}
    {#1}
\titleformat{\section}
  {\normalfont\Large\sffamily\bfseries}{\thesection\space\space}
  {1ex}
  {#1}
\titleformat{\subsection}
  {\normalfont\large\sffamily}{\thesubsection\space\space}
  {1ex}
  {#1}

% =======================
% = Cabeceras de página =
% =======================
\usepackage[]{fancyhdr}
\usepackage[]{emptypage}
\fancypagestyle{plain}{%
    \fancyhf{}%
    \renewcommand{\headrulewidth}{0pt}
    \renewcommand{\footrulewidth}{0pt}
}
\fancypagestyle{tfg}{%
    \fancyhf{}%
    \renewcommand{\headrulewidth}{0pt}
    \renewcommand{\footrulewidth}{0pt}
    \fancyhead[LE]{{\normalsize\color{USred}\bfseries\thepage}\quad
                    \small\textsc{\MakeLowercase{\maintitle}}}
    \fancyhead[RO]{\small\textsc{\MakeLowercase{\rightmark}}%
                    \quad{\normalsize\bfseries\color{USred}\thepage}}%
                    }

% =============================
% = El documento empieza aquí =
% =============================
\begin{document}


\maketitle

\frontmatter
\tableofcontents

\mainmatter

%========================
\chapter*{English Abstract}
\addcontentsline{toc}{chapter}{English Abstract}
\markright{English Abstract}



\begin{otherlanguage}{english}
    Analysis of subcell phenomenology in the Yee algorithm using the FDTD method...
\end{otherlanguage}



\chapter{Introduction to the FDTD method.}

The numerical FDTD method (``finite difference time domain") enables the equations in partial derivatives resolution. As a first step, this method will be applied to one of the simplest equations, the one dimensional wave equation:
\begin{equation}
\frac{\partial^2 u}{\partial t^2}= c^2 \frac{\partial^2 u}{\partial x^2}.
\end{equation}
The analytic solution of this equation is known and it is just the combination of two functions $F$ and $G$ in the following way:
\begin{equation}
u(x,t)= F(x+ct)+G(x-ct).
\end{equation}
\indent Now that we know this, finite differences will be computed. First, the following function will be expanded as a Taylor series, $u(x,t_n)$, about the point $x_i$ to point $x_i+\Delta x$, keeping time fixed [Allen Taflove,Susan C.Hagness]:
\begin{equation}
u(x_i+\Delta x)|_{t_n} = u|_{x_i,t_n} + \Delta x \frac{\partial u}{\partial x} \bigg |_{x_i,\: t_n} + \frac{\left(\Delta x\right)^2}{2}\frac{\partial^2 u}{\partial x^2}\bigg |_{x_i,\: t_n} +\frac{\left(\Delta x\right)^3}{3!}\frac{\partial^3 u}{\partial x^3}\bigg |_{x_i,\: t_n} + \frac{\left(\Delta x\right)^4}{4!}\frac{\partial^4 u}{\partial x^4}\bigg |_{\xi_1,\: t_n},
\end{equation} 
where the last term is the error term and point $\xi_1$ is located in the interval $(x_i, x_i+\Delta x)$. If series expansion is carried about point $x_i - \Delta x$, and both expresions are added, it remains:
\begin{equation}
u(x_i+\Delta x) \bigg |_{t_n}+u(x_i-\Delta x) \bigg |_{t_n}=2u|_{x_i,t_n}+\left(\Delta x\right)^2\frac{\partial^2 u}{\partial x^2}\bigg |_{x_i,\: t_n} + O\left[ \left( \Delta x\right)^2\right],
\end{equation}
the last term being the error...


\chapter{Maxwell's equations and the Yee algorithm.}

Maxwell's equations in their general form are written in three dimensions as...



\section{One dimensional Maxwell's equations.}
One dimensional Maxwell's equations, in free space, are reduced to:

\begin{align}
& \frac{\partial E_x}{\partial t}=-\frac{1}{\epsilon_0}  \frac{\partial H_y}{\partial z}, \\
& \frac{\partial H_y}{\partial t}=-\frac{1}{\mu_0} \frac{\partial E_x}{\partial z}.
\end{align}

These equations are those of a plane wave propagating in the $z$ direction with the electric field in the $x$ direction and the magnetic field oriented in the $y$ direction. Using the central difference approximation in the spatial and time derivatives, the following form of the equations is obtanied:

\begin{align}
 \frac{E_x^{n+\frac{1}{2}}(k)-E_x^{n-\frac{1}{2}}(k)}{\Delta t}=-\frac{1}{\epsilon_0}\frac{H_y^n \left(k+\frac{1}{2}\right) - H_y^n\left(k-\frac{1}{2}\right)}{\Delta x} \\
\frac{H_y^{n+1} \left(k+\frac{1}{2}\right) -H_y^{n}\left( k +\frac{1}{2}\right)}{\Delta t}=-\frac{1}{\mu_0}\frac{E_x^{n+\frac{1}{2}} \left(k+1\right) - E_x^{n+\frac{1}{2}}\left(k\right)}{\Delta x}.
\end{align}

...
\subsection{Propagation in a dielectric medium.}


\subsection{Propagation in a lossy dielectric medium.}



\chapter{Stability in the FDTD method.}
Courant condition.


\chapter{Source types (Hard source/Soft Source).}


















%\chapter{Los enunciados}

%\section{Teoremas y demostraciones}


%\begin{theorem}[Euclides]\label{thm:th1}
   % Esto es un Teorema. Se numeran a partir del 1 en cada capítulo. Como son importantes, %tienen un cuadrado rojo al principio. Llevan letra cursiva.
%\end{theorem}

%\begin{proof}
%    Esto es la demostración. Al final de la demostración se puede ver un cuadrado rojo %similar al de los teoremas. Las demostraciones no llevan letra cursiva.
%\end{proof}


%\begin{definition}\label{def:1}
%    Esto es una definición. Las definiciones son importantes; también llevan un %cuadradito rojo.
%\end{definition}


%\subsection{Otros enunciados}


%\begin{remark}
%    Esto es una observación, que dice que $e=mc^{2}$. Como las observaciones no son %importantes, no llevan cuadrado rojo, y el tipo de letra no es cursiva.
%\end{remark}


%\begin{proof}
    %Si la demostración acaba en una fórmula, para poner el cuadrado rojo a la altura de %la última formula, hay que usar la orden \verb|\qedhere|, como en este caso:
%    \[
%        e=mc^{2}.\qedhere
%   \]

%\end{proof}


%\begin{corollary}\label{cor:1}
  %  Esto es un corolario.
%\end{corollary}

%\begin{proposition}\label{pro:1}
%   Esto es una proposición.
%\end{proposition}

%\begin{lemma}[Gauss]\label{lem:1}
    %Esto es un lema.
%\end{lemma}


%\backmatter

%\bibliographystyle{acm}
% \biliography{miarchivo} % -> lee miarchivo.bib
%===============


\end{document}